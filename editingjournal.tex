\documentclass{article}
\usepackage[utf8]{inputenc}

\title{Editing Journal For My Poker Solver}
\author{Eric Wang}
\date{Start: December 23, 2023}

\begin{document}
\maketitle
\newpage

\section{Code Structural Diagram} 

\noindent PreflopHands Class
\begin{itemize}
    \item contains a 75x75 array of the 
preflop ranking of every starting hand
in poker
\end{itemize}

\noindent Card Class
\begin{itemize}
    \item each card is given a number and
a suit
    \item each card also has a unique 
cardValue to make it simpler to distinguish
different cards. the card value is the number
times 5 + the suit, suits are 1-4
\end{itemize}

\noindent Hand Class
\begin{itemize}
    \item a hand has 2 cards
    \item additionally, a hand stores the
information about whether or not it is suited
\end{itemize}

\noindent Deck Class
\begin{itemize}
    \item a deck is initialized as a list of 
all 52 cards, shuffled randomly
\end{itemize}

\noindent Board Class
\begin{itemize}
    \item a board has a list of all the cards 
on it
    \item it comes with methods based on a 
deck to deal the next cards in accordance to
the game
\end{itemize}

\noindent Player Class
\begin{itemize}
    \item a player has a hand, a stack, and 
a position for which order they take their 
turn
    \item a player also comes with the actions
fold, check, raise quarter-pot, raise half-pot, 
raise three-quarter-pot, raise pot, and reraise
\end{itemize}

\noindent Round Class
\begin{itemize}
    \item each round has a deck, from which 
each player is dealt a hand
    \item each round also has a board and a 
pot, and each round implements the normal 
circle of betting. 
    \item after every circle, the flop, turn
and river are dealt and eventually the result
of the game is reached, meaning that each 
player's hand is ranked and a winner is chosen
\end{itemize}

\noindent Game Class
\begin{itemize}
    \item a game has a number of players, the 
starting stack for each player, as well as the
number of rounds that will be played.
\end{itemize}

\section{Dec. 23, 2023}

Brainstorm:
\begin{itemize}
    \item first, implement a program that 
calculates the odds of every hand winning
without any other information, preflop, 
postflop, after the turn and river. This 
way, it will make it easier later on to 
categorize hands by strength to ease the 
decision making process for the machine
    \item then, we can implement a neural
net to play the actual hands, passing in
hand strength and betting history as 
parameters
    \item for this, we will have to learn 
python as it is likely the easiest language
to do this in
    \item after implementing the first 
neural net, we can see how it does playing
against me/itself and figure out how to 
best train the program going forward.
\end{itemize} 

\noindent Report:
\begin{itemize}
    \item hello world in python!
\end{itemize} 

\section{Dec. 24, 2023}

Plan:
\begin{itemize}
    \item start hardcoding all the preflop
hands and setting up the infrastructure of 
my code
    \item figure out how classes work and
define a hand class, card class, maybe a 
board class
\end{itemize}

\noindent
Notes:
\begin{itemize}
    \item implemented hand, card, and deck 
classes
    \item hardcoded a small portion of the 
preflop hands. look to finish them tomorrow
and then start working on implementing the 
rules of poker and who has the best hand
\end{itemize}

\section{Dec. 25, 2023}
Notes:
\begin{itemize}
    \item finished hardcoding the preflop 
hands
    \item working on how to select the 
winning hand given a board and each player's 
hand
\end{itemize}

\noindent
Notes:
\begin{itemize}
    \item Preflop hands have been implemented.
    \item We need a way to calculate odds 
postflop, as precalculation would be $10^{12}$ 
and take way too long.
    \item Instead, we will just take the hand 
we currently have and compare it to every other
possible hand as well as every other possible 
runout, which is around $10^7$.
    \item So now, we need to implement picking 
a winning hand from a board and a list of hands
    \item we also need to be able to run the
actual game of poker
    \item wrote up a general outline for all of 
the classes that are made
\end{itemize}

\section{Dec. 27, 2023}
Notes:
\begin{itemize}
    \item took a break yesterday, as I had some 
other arrangements
    \item spent today implementing choosing a 
winning hand after the board has been dealt
    \item also finished implementing the round
class
\end{itemize}

\section{Dec. 28, 2023}
Notes:
\begin{itemize}
    \item spent some time on researching how
to construct the actual machine learning parameters
of the code
    \item the structural aspect and the actual game
is approaching completion, but the neural network
related stuff seems to be posing a few problems
    \item how am I going to determine which 
modification to make with the regret adjusting? And 
how am I going to determine whether or not the 
program is improving?
\end{itemize}

\section{Jan. 2, 2024}
Notes:
\begin{itemize}
    \item took a break, back from some celebrations
as well as recovering from being sick
    \item spent the day on the Micrograd_Follow_Along
as well as neuralnetworknotes.tex, watching videos to 
study neural networks
\end{itemize}

\end{document}